\documentclass[11pt]{article}

% ====== minimal arXiv-friendly packages ======
\usepackage[T1]{fontenc}
\usepackage[utf8]{inputenc}
\usepackage{lmodern}
\usepackage{microtype}
\usepackage{amsmath, amssymb, amsthm}
\usepackage{mathtools}
\usepackage{booktabs}
\usepackage{hyperref}
\usepackage{geometry}
\geometry{margin=1in}

% ====== theorem environments (lightweight) ======
\newtheorem{theorem}{Theorem}
\newtheorem{definition}{Definition}

% ====== handy macros ======
\newcommand{\Z}{\mathbb{Z}}
\newcommand{\cfsub}{\mathrm{cf\_sub}}
\newcommand{\cfneg}{\mathrm{cf\_neg}}

\title{$\varphi$-NTT: A Carry-Free Transform on $\Z_{10}^{B}$ with Hierarchical Wavelet Structure}
\author{morcb13-bit}
\date{}

\begin{document}
\maketitle

% Reference header as a footnote (keeps paper clean)
\begingroup
\renewcommand\thefootnote{}\footnote{
Reference implementation (frozen): \url{https://github.com/morcb13-bit/Artemis-B13-Archive/tree/main/tutorial} @ \texttt{[commit SHA here]}.
Source: Handoff notes v2--v12 (2026-02-25).
Policy: No HTML creation, no B=4/5 recomputation, no new experiments, no visualization additions.
Next task: finalize \LaTeX\ for arXiv.
}
\addtocounter{footnote}{-1}
\endgroup

\begin{abstract}
We introduce $\varphi$-NTT (phi Number Theoretic Transform), a transform defined over the carry-free direct product group $\Z_{10}^{B}$ with coefficients in the golden-ratio integer ring $\Z[\varphi] = \Z[\sqrt{5}]$. The transform achieves exact integer arithmetic with no floating-point error and no inter-stage twiddle factors. The $B$-stage construction yields a $2^{B}$ channel filterbank with a hierarchical wavelet structure analogous to Haar MRA. We establish a Spectral Concentration Quasi-Theorem: for signals with dominant period $P \approx 2\cdot 10^{b}$, the detail channel at level $b$ captures energy fraction $\alpha_b \ge C(1-\varepsilon)^{B-1}$. Empirically, the top 3 channels cover $\ge 97\%$ of total energy for $B=3..5$.
\end{abstract}

\section{Introduction}
The Fast Fourier Transform (FFT) and its variants are among the most widely deployed algorithms in signal processing and communications. Standard FFT operates on cyclic groups $\Z/(N)$ and relies on complex-valued twiddle factors, which introduce floating-point error and prevent exact integer arithmetic.

In this paper, we introduce the $\varphi$-NTT (phi Number Theoretic Transform), a transform defined over the direct product group
$\Z_{10}^{B} = \Z_{10}\times \cdots \times \Z_{10}$ --- the group of $B$-digit decimal representations with carry-free (digit-wise modular) arithmetic. The key structural features of $\varphi$-NTT are:
\begin{itemize}
\item \textbf{Exact integer arithmetic:} all coefficients lie in the ring $\Z[\varphi]=\Z[\sqrt{5}]$ (golden-ratio integer ring); no floating-point error.
\item \textbf{Zero inter-stage twiddle factors:} the tensor-product theorem eliminates all cross-stage phase corrections over $B$ recursive stages.
\item \textbf{$2^B$ channel filterbank:} the $B$-stage transform decomposes a length-$10^B$ signal into $2^B$ channels, labeled by strings over $\{T,U\}$, where $T$ denotes a $C_5$-twisted low-pass projection and $U$ denotes a $\Z_2$ high-pass (parity) projection.
\end{itemize}

The carry-free arithmetic setting --- where each decimal digit evolves independently --- naturally gives rise to a hierarchical wavelet structure analogous to the Haar multiresolution analysis (MRA), but defined over the ring $\Z[\varphi]$ and indexed by decimal digits rather than binary positions. We formalize this analogy and prove a Spectral Concentration Quasi-Theorem: for signals with dominant period $P \approx 2\cdot 10^{b}$, the detail channel at level $b$ captures a fraction $\alpha_b \ge C(1-\varepsilon)^{B-1}$ of total signal energy. Empirically, the top 3 channels cover $\ge 97\%$ of energy for $B=3..5$.

The remainder of this paper is organized as follows. Section~2 defines $\varphi$-NTT and the convolution theorem on $\Z_{10}^{B}$. Section~3 develops the hierarchical wavelet interpretation. Section~4 states and sketches the proof of the quasi-theorem. Section~5 presents numerical experiments for $B=1..5$. Section~6 discusses open problems, including the exact orthogonality of $T/U$ projections and the determination of constants $C$ and $\varepsilon$. Section~7 concludes.

\section{$\varphi$-NTT on $\Z_{10}^{B}$}

\subsection{Carry-Free Group Structure}
Let $B$ be a positive integer and $N = 10^B$. We represent each index $n \in \{0,\dots,N-1\}$ by its decimal digit expansion
\[
n = d_0(n) + 10d_1(n) + \cdots + 10^{B-1}d_{B-1}(n), \quad d_i(n)\in\{0,\dots,9\}.
\]
The carry-free group $\Z_{10}^{B}$ is the direct product $\Z_{10}\times\cdots\times\Z_{10}$ ($B$ copies), with addition defined digit-wise:
\[
(n \oplus m)_i = (d_i(n) + d_i(m)) \bmod 10.
\]
This differs fundamentally from ordinary modular arithmetic on $\Z/(10^B)\Z$, where addition carries across digit boundaries. The two groups coincide only at $B=1$; for $B\ge 2$ they are non-isomorphic, and the standard cyclic convolution theorem does not apply to $\Z_{10}^{B}$.

Carry-free subtraction and negation are defined analogously:
\[
\cfsub(n,m)_i = (d_i(n) - d_i(m)) \bmod 10,\qquad
\cfneg(n)_i = (-d_i(n)) \bmod 10.
\]
All arithmetic in $\varphi$-NTT operates over the ring $\Z[\varphi]=\Z[\sqrt{5}]$, where $\varphi=(1+\sqrt5)/2$. Elements are pairs $(a,b)\in\Z^2$ representing $a+b\varphi$, with multiplication
\[
(a,b)\cdot(c,d) = (ac+bd,\ ad+bc+bd).
\]
This eliminates floating-point error entirely.

\subsection{Definition of $\varphi$-NTT}
For $B=1$ (length $N=10$), the transform decomposes the input $x:\Z_{10}\to\Z[\varphi]$ into two channels via the isomorphism $\Z_{10}\cong \Z_2\times \Z_5$ (Chinese Remainder Theorem):
\[
\text{$T$-channel: } X_T[k] = n10\_T(x)[k]\quad (C_5\text{-twisted DFT projection}),\qquad
\text{$U$-channel: } X_U[k] = n10\_U(x)[k]\quad (\Z_2\text{ parity projection}).
\]
The $C_5$-twisted DFT uses basis elements $\mathrm{cos5\_zphi}(m)\in\{(2,0),(-1,1),(0,-1)\}$ --- three distinct values in $\Z[\varphi]$, with no irrational twiddle factors at runtime. The $\Z_2$ projection uses the parity kernel $U_1[k]\in\{(0,0),(\pm1,0),(0,\pm1)\}$. Explicit tables are given in Appendix~A.

For general $B$, the transform is defined by a $B$-stage tensor product. At each stage $b\in\{0,\dots,B-1\}$, the input is projected onto either the $T$-channel (low-pass, $C_5$ side) or the $U$-channel (high-pass, $\Z_2$ side). This yields $2^B$ output channels, each labeled by a string over $\{T,U\}^B$:
\[
bn\_forward\_flat(x,B) \to [ch_s : s\in\{T,U\}^B],\quad \text{each } ch_s:\Z_{10}^{B}\to\Z[\varphi].
\]
The inverse transform $bn\_inverse\_flat$ reconstructs $x$ exactly, with scale factor $40^B$ (no convolution) or $80^B=40^B\cdot 2^B$ (after channel-domain convolution). Both scale factors divide exactly in $\Z[\varphi]$, so reconstruction is lossless over the integers.

A key structural property is that no inter-stage twiddle factors appear: the tensor-product theorem over $\Z_{10}^{B}$ guarantees that all cross-stage phase corrections vanish across all $B$ recursive stages.

\subsection{Convolution Theorem (Carry-Free)}
Define carry-free convolution of $x,h:\Z_{10}^{B}\to\Z[\varphi]$ as
\[
(x\star h)[n] = \sum_m h[m]\cdot x[\cfsub(n,m)],\quad n\in\Z_{10}^{B}.
\]
\begin{theorem}[Carry-Free Convolution]
Let $F_B$ denote the $\varphi$-NTT forward transform. Then $F_B$ diagonalizes carry-free convolution on $\Z_{10}^{B}$: the channel-domain product
\[
Y_s[k] = X_s[k]\cdot H_s[k]\quad\text{(with the $T/U$ cross-coupling rule)}
\]
recovers $x\star h$ upon inverse transform, up to the scale factor $80^B$. The $T/U$ cross-coupling at each stage is:
\[
2Y_T[k] = X_T[k]H_T[k] - \mathrm{SIN2}\, X_U[k]H_U[k],\qquad
2Y_U[k] = X_T[k]H_U[k] + X_U[k]H_T[k],
\]
where $\mathrm{SIN2}=(3,-1)\in\Z[\varphi]\approx 1.382 = 4\sin^2(2\pi/10)$. This has been verified for $B=1..5$ against direct computation.
\end{theorem}

Similarly, carry-free cross-correlation reduces to convolution with the carry-free-reversed kernel $h_{\mathrm{rev}}[m] = h[\cfneg(m)]$.

\section{Hierarchical Wavelet Interpretation}
The recursive $T/U$ decomposition induces a hierarchical structure on $\Z_{10}^{B}$ that parallels multiresolution analysis (MRA).

\subsection{$T/U$ as Projection-Like Operators}
At each stage $b\in\{0,\dots,B-1\}$, the $\varphi$-NTT applies one of two projection-like operators to digit $d_b$:
\[
T_b:\ \text{low-pass side }(C_5\text{-twisted DFT projection})\ \text{extracts slowly-varying component across } d_b\in\{0,\dots,9\},
\]
\[
U_b:\ \text{high-pass side }(\Z_2\text{ parity projection})\ \text{extracts alternating component } (-1)^{d_b}.
\]
Together, $T_b$ and $U_b$ partition the information in digit $b$ analogously to a two-channel filter bank. We use \emph{projection-like} deliberately: while numerical experiments confirm that energy is partitioned across channels without loss (exact reconstruction holds for $B=1..5$), strict algebraic orthogonality $T_b\perp U_b$ has not yet been proved and is left as an open problem (Section~6).

The $B$-stage tensor product applies these operators independently at each digit position, yielding $2^B$ channels:
\[
ch_s = \left(\bigotimes_{b=0}^{B-1} P_b^{s_b}\right)(x),\quad s\in\{T,U\}^B,
\]
where $P_b^T=T_b$ and $P_b^U=U_b$. Full signal energy is distributed across all $2^B$ channels, and exact reconstruction is guaranteed by $bn\_inverse\_flat$.

\subsection{Binary Tree Interpretation}
The label set $\{T,U\}^B$ is in natural bijection with the set of root-to-leaf paths in a complete binary tree of depth $B$. Reading the label from left (most significant digit, $b=B-1$) to right (least significant, $b=0$), each leaf corresponds to one channel. The channel $TT\cdots T$ (all $T$, leftmost leaf) captures the coarsest approximation; $UU\cdots U$ (all $U$, rightmost leaf) captures the finest detail --- a checkerboard pattern across all digit positions simultaneously.

The structure is analogous to Haar MRA but not identical; key correspondences are summarized in Table~\ref{tab:haar}.

\begin{table}[h]
\centering
\caption{Correspondences between Haar MRA and $\varphi$-NTT.}
\label{tab:haar}
\begin{tabular}{@{}lll@{}}
\toprule
Property & Haar MRA & $\varphi$-NTT \\
\midrule
Signal length & $2^B$ & $10^B$ \\
Channels & $2^B$ & $2^B$ \\
Approximation coeff. & scaling function & $TT\cdots T$ channel \\
Detail coeff. & wavelet function & $T\cdots TU_bT\cdots T$ channels \\
Arithmetic & real ($\pm 1$ basis) & $\Z[\varphi]$ basis \\
Orthogonality & exact & conjectured \\
\bottomrule
\end{tabular}
\end{table}

\subsection{Energy Localization Mechanism}
Smooth signals vary slowly across all digit positions. The $U_b$ operator measures parity alternation $(-1)^{d_b}$; if $x$ changes little between even and odd values of $d_b$, the $U_b$ inner product is small. Accumulating this across $B$ stages, the $UU\cdots U$ channel --- requiring large alternation at \emph{every} digit simultaneously --- is driven toward zero. This explains the empirical observation ($B=3..5$) that $UU\cdots U$ carries $<0.01\%$ of total energy for smooth signals.

Signals with dominant period $P\approx 2\cdot 10^b$ exhibit strong alternation specifically at digit $b$, while other digits vary slowly. The $U_b$ projection captures the bulk of oscillatory energy, while $T$ projections at other stages pass the slowly varying envelope. Thus $T\cdots T[U_b]T\cdots T$ becomes dominant. For $\sin(2\pi n/100)$ with $P=100\approx 2\cdot 10^1$, this predicts $TUT\cdots T$ as dominant --- consistent with experiments for $B=3..5$ (86\%, 69\%, 68\%).

\section{Spectral Concentration}
We formalize the observed concentration phenomenon as a quasi-theorem.

\subsection{Statement}
We work under two assumptions on the input signal $x:\Z_{10}^{B}\to\Z[\varphi]$ (real-valued, i.e., second component identically 0):
\begin{itemize}
\item \textbf{(A1) Digit-wise smoothness.} For each digit position $b'\ne b$, the variation of $x$ across $d_{b'}$ is small relative to the variation across $d_b$:
\[
|\mathrm{variation}(x,b')|\le \varepsilon\, |\mathrm{variation}(x,b)|,\quad b'\ne b,
\]
for some $0<\varepsilon<1$. Here $\mathrm{variation}(x,b)$ measures the mean absolute change in $x$ as digit $b$ steps through $\{0,\dots,9\}$.
\item \textbf{(A2) Dominant period.} The signal has dominant period $P\approx 2\cdot 10^b$ for some $b\in\{0,\dots,B-1\}$, so that $d_b$ is the primary source of oscillation.
\end{itemize}

\begin{theorem}[$\varphi$-NTT Spectral Concentration Quasi-Theorem]
Under assumptions (A1) and (A2), there exist constants $C>0$ and $0<\varepsilon<1$ (depending on the signal class) such that the energy fraction captured by the level-$b$ detail channel satisfies
\[
\alpha_b = \frac{E[ch_{T\cdots TU_bT\cdots T}]}{E_{\mathrm{total}}}\ \ge\ C(1-\varepsilon)^{B-1}.
\]
Furthermore, the top 3 channels (by energy) jointly satisfy
\[
\alpha_b + \alpha_{b,2} + \alpha_{b,3}\ \ge\ 1-\delta,
\]
for a small residual $\delta$, empirically $\delta<0.03$ for $B=3..5$.
\end{theorem}

We call this a \emph{quasi-theorem} because $C$ and $\varepsilon$ are characterized implicitly through (A1)--(A2) rather than in closed form, and because strict algebraic orthogonality of the $T/U$ decomposition --- while consistent with numerical evidence --- remains unproved (Section~6).

\subsection{Interpretation}
The bound $C(1-\varepsilon)^{B-1}$ has transparent structure: $\varepsilon$ measures digit-level leakage from the dominant digit $b$ into the other $B-1$ digits, while $(1-\varepsilon)^{B-1}$ is the product of per-stage suppression factors. The observed $\alpha_b$ decreases from 86\% ($B=3$) to $\approx 68\%$ ($B=4,5$) as more stages accumulate leakage, while top-3 coverage stays $\ge 97\%$ because nearest channels absorb most leakage.

\subsection{Proof Sketch}
\textbf{Step 1} (Digit smoothness $\Rightarrow$ $U$ energy bound). By (A1), for each $b'\ne b$, the $U_{b'}$ projection (parity alternation at digit $b'$) has small inner product with $x$, yielding $E[U_{b'}]\le \varepsilon^2 E[U_b]$.

\textbf{Step 2} (Dominant digit $\Rightarrow$ $U_b$ lower bound). By (A2), $x$ correlates strongly with $(-1)^{d_b}$, the kernel of $U_b$, yielding $E[U_b]\ge C_0 E_{\mathrm{total}}$ for some $C_0>0$.

\textbf{Step 3} (Tensor product $\Rightarrow$ multiplicative separation). The channel $T\cdots TU_bT\cdots T$ receives energy from $U_b$ at stage $b$ and $T$ at other stages; $U_{b'}$ at other stages is suppressed by $\varepsilon$ per stage.

\textbf{Step 4} (Normalization $\Rightarrow$ lower bound). Summing energy across all $2^B$ channels yields $\alpha_b \ge C(1-\varepsilon)^{B-1}$ with $C$ absorbing $C_0$ and pass-through factors.

\subsection{Empirical Alignment}
For $B=3..5$ and test signal $\sin(2\pi n/100)$ with $P=100\approx 2\cdot 10^1$ ($b=1$):
\begin{center}
\begin{tabular}{@{}lll@{}}
\toprule
$B$ & $\alpha_b$ (observed) & Lower bound structure \\
\midrule
3 & 0.864 & $C(1-\varepsilon)^2$ with $\varepsilon\approx 0.07$ \\
4 & 0.687 & $C(1-\varepsilon)^3$ \\
5 & 0.682 & $C(1-\varepsilon)^4$ \\
\bottomrule
\end{tabular}
\end{center}
Top-3 channels cover $\ge 97\%$ and $UU\cdots U$ carries $<0.01\%$ in all experiments.

\section{Numerical Experiments}

\subsection{Experimental Setup}
All experiments use the reference implementation \texttt{phi\_carry\_free.py}, operating entirely in exact integer arithmetic over $\Z[\varphi]$. No floating-point operations occur in the transform itself; \texttt{zval()} is used only for energy measurement in post-processing.

We test $B=1..5$ (signal lengths $N=10$ to $100{,}000$) with a low-frequency test signal
\[
x[n] = \sin(2\pi n/100) + \sin(2\pi n/P_2) + \text{noise},
\]
where $P_2$ is a secondary period scaled to $N$ and noise is a small integer perturbation. This signal class satisfies assumptions (A1)--(A2) with dominant period $P=100\approx 2\cdot 10^1$ ($b=1$).

Correctness is verified by (i) direct comparison of \texttt{phi\_conv\_carry\_free} against the reference $O(N^2)$ carry-free convolution, and (ii) round-trip test with a delta kernel $h=\delta_0$, confirming exact reconstruction for $B=1..5$.

\subsection{Channel Energy Distribution}
Energy fractions are computed as
\[
\frac{E[ch]}{E_{\mathrm{total}}},\quad
E[ch] = \frac{1}{N}\sum_k \bigl(\mathrm{zval}(ch[k])\bigr)^2\quad\text{(real-valued signal assumption)}.
\]

\begin{table}[h]
\centering
\caption{Channel energy distribution (low-frequency test signal).}
\label{tab:energy}
\begin{tabular}{@{}lllllllll@{}}
\toprule
$B$ & \#1 ch & $\alpha_1$ & \#2 ch & $\alpha_2$ & \#3 ch & $\alpha_3$ & Top-3 & $UU\cdots U$ \\
\midrule
3 & TUT   & 86.4\% & TTT    & 11.1\% & UTT    & 2.3\%  & 99.8\% & $<0.01\%$ \\
4 & TUTT  & 68.7\% & TTUT   & 17.7\% & TTTT   & 11.7\% & 98.1\% & $<0.01\%$ \\
5 & TUTTT & 68.2\% & TTTUT  & 17.7\% & TTTTT  & 11.7\% & 97.6\% & $<0.01\%$ \\
\bottomrule
\end{tabular}
\end{table}

Three observations are immediate. (i) The dominant channel is consistently of the form $T\cdots TU_1T\cdots T$ (U only at digit $b=1$), matching the prediction for $P\approx 2\cdot 10^1$. (ii) Top-3 channels jointly cover $\ge 97\%$, consistent with $\delta<0.03$. (iii) $UU\cdots U$ carries negligible energy, consistent with checkerboard suppression.

\begin{table}[h]
\centering
\caption{SNR vs.\ retained channels (low-frequency signal).}
\label{tab:snr}
\begin{tabular}{@{}lllll@{}}
\toprule
$B$ & 1 ch & 2 ch & 4 ch & All ch \\
\midrule
3 & 9.5 dB & 15.4 dB & 28.5 dB & $\infty$ (exact) \\
4 & 5.3 dB & 9.5 dB  & 20.5 dB & $\infty$ (exact) \\
5 & 5.2 dB & 9.3 dB  & 18.5 dB & $\infty$ (exact) \\
\bottomrule
\end{tabular}
\end{table}

\subsection{Reconstruction Accuracy}
Exact reconstruction is guaranteed by the algebraic structure of $\varphi$-NTT, not numerical precision. Scale factors $40^B$ (round-trip) and $80^B$ (after convolution) divide exactly in $\Z[\varphi]$, verified for $B=1..5$ by assertion checks. No floating-point rounding occurs.

\subsection{Computational Structure}
$\varphi$-NTT avoids two complexity sources common in FFT: no inter-stage twiddle factors (all phase corrections vanish by the tensor-product theorem over $\Z_{10}^{B}$), and integer-only arithmetic (divisions are only final exact scale normalizations). The transform operates on length-$10^B$ signals with $2^B$ channels in $O(N\cdot B)$ integer operations per stage. Detailed complexity analysis and FFT comparisons are reserved for future work.

\section{Discussion}

\subsection{Open Problems}
\textbf{(P1) Orthogonality of $T/U$ projections.} Numerical evidence shows energy partition and exact reconstruction, but strict algebraic orthogonality $T_b\perp U_b$ over $\Z[\varphi]$ remains unproved.

\textbf{(P2) Closed-form determination of $C$ and $\varepsilon$.} Constants in the quasi-theorem are characterized implicitly via (A1)--(A2). Closed forms in terms of period $P$, base 10, and $\Z[\varphi]$ basis would make bounds signal-class independent.

\textbf{(P3) Connection to carry-based convolution (overlap-add).} Current work is restricted to carry-free convolution on $\Z_{10}^{B}$. Bridging to ordinary cyclic convolution on $\Z/(10^B)\Z$ via overlap-add is non-trivial due to $\Z_{10}^B \not\cong \Z/(10^B)\Z$ for $B\ge 2$.

\subsection{Relation to Existing Frameworks}
\textbf{vs.\ FFT.} FFT diagonalizes cyclic convolution on $\Z/(N)$ using complex twiddles and floating point; $\varphi$-NTT diagonalizes a different algebra (carry-free convolution on $\Z_{10}^{B}$) over $\Z[\varphi]$, enabling exact integer arithmetic.

\textbf{vs.\ Haar MRA.} The binary-tree structure is analogous but inexact: Haar uses $\mathbb{R}$ and $\pm1$ basis on $\Z/(2^B)\Z$, while $\varphi$-NTT uses $\Z[\varphi]$ basis on $\Z_{10}^{B}$.

\textbf{vs.\ NTT.} Classical NTT replaces complex roots with modular roots but retains cyclic $\Z/(N)$ structure. $\varphi$-NTT uses direct-product $\Z_{10}^{B}$ and 
