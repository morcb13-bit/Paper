% phi-NTT paper — arXiv minimal template
% Source: phi-ntt-paper.md (引継書 v2〜v12, 2026-02-25)
% Reference impl: https://github.com/morcb13-bit/Artemis-B13-Archive/tree/main/tutorial
% Commit SHA: [PAPER_COMMIT_SHA]  (fill in before submission)

\documentclass[11pt,a4paper]{article}

% ── Packages ──────────────────────────────────────────────────────────────────
\usepackage{amsmath,amssymb,amsthm}
\usepackage{mathtools}
\usepackage{booktabs}
\usepackage{hyperref}
\usepackage{microtype}
\usepackage[margin=1in]{geometry}

% ── Theorem environments ───────────────────────────────────────────────────────
\newtheorem{theorem}{Theorem}[section]
\newtheorem{definition}[theorem]{Definition}
\newtheorem{lemma}[theorem]{Lemma}
\newtheorem{remark}[theorem]{Remark}

% ── Notation shorthands ────────────────────────────────────────────────────────
\newcommand{\ZB}{\mathbb{Z}_{10}^{B}}
\newcommand{\Zte}{\mathbb{Z}_{10}}
\newcommand{\Zphi}{\mathbb{Z}[\varphi]}
\newcommand{\cfadd}{\oplus}
\newcommand{\cfsub}{\ominus}
\newcommand{\cfneg}{\ominus}
\newcommand{\phiNTT}{\varphi\text{-NTT}}

% ── Title block ───────────────────────────────────────────────────────────────
\title{$\varphi$-NTT: A Carry-Free Transform on $\mathbb{Z}_{10}^{B}$
       with Hierarchical Wavelet Structure}

\author{%
  % Author name(s) — fill in before submission
  [Author]\\
  \small [Affiliation]\\
  \small \texttt{[email]}
}

\date{\today}

% ══════════════════════════════════════════════════════════════════════════════
\begin{document}
\maketitle

% ── Abstract ──────────────────────────────────────────────────────────────────
\begin{abstract}
We introduce $\varphi$-NTT (phi Number Theoretic Transform), a transform defined
over the carry-free direct product group $\mathbb{Z}_{10}^{B}$ with coefficients
in the golden-ratio integer ring $\Zphi = \mathbb{Z}[\sqrt{5}]$.
The transform achieves exact integer arithmetic with no floating-point error and
no inter-stage twiddle factors.
The $B$-stage construction yields a $2^{B}$ channel filterbank with a hierarchical
wavelet structure analogous to Haar MRA\@.
We establish a \emph{Spectral Concentration Quasi-Theorem}: for signals with
dominant period $P \approx 2 \cdot 10^{b}$, the detail channel at level $b$
captures energy fraction $\alpha_{b} \ge C(1-\varepsilon)^{B-1}$.
Empirically, the top~3 channels cover $\ge 97\%$ of total energy for
$B = 3, 4, 5$.
\end{abstract}

% ══════════════════════════════════════════════════════════════════════════════
\section{Introduction}
\label{sec:intro}

The Fast Fourier Transform (FFT) and its variants are among the most widely
deployed algorithms in signal processing and communications.
Standard FFT operates on cyclic groups $\mathbb{Z}/(N)$ and relies on
complex-valued twiddle factors, which introduce floating-point error and prevent
exact integer arithmetic.

In this paper, we introduce the \emph{$\varphi$-NTT} (phi Number Theoretic
Transform), a transform defined over the direct product group
$\mathbb{Z}_{10}^{B} = \Zte \times \cdots \times \Zte$
(the group of $B$-digit decimal representations with carry-free,
digit-wise modular arithmetic).
The key structural features of $\varphi$-NTT are:
\begin{itemize}
  \item \textbf{Exact integer arithmetic.}
        All coefficients lie in the ring $\Zphi = \mathbb{Z}[\sqrt{5}]$
        (golden-ratio integer ring); no floating-point error occurs.
  \item \textbf{Zero inter-stage twiddle factors.}
        The tensor-product theorem eliminates all cross-stage phase corrections
        across all $B$ recursive stages.
  \item \textbf{$2^{B}$ channel filterbank.}
        The $B$-stage transform decomposes a length-$10^{B}$ signal into $2^{B}$
        channels, labeled by binary strings over $\{T, U\}^{B}$, where $T$
        denotes a $C_{5}$-twisted low-pass projection and $U$ denotes a $\mathbb{Z}_{2}$
        high-pass (parity) projection.
\end{itemize}

The carry-free arithmetic setting---where each decimal digit evolves
independently---naturally gives rise to a hierarchical wavelet structure
analogous to Haar multiresolution analysis (MRA), but defined over the ring
$\Zphi$ and indexed by decimal digits rather than binary positions.
We formalize this analogy and prove a Spectral Concentration Quasi-Theorem:
for signals with dominant period $P \approx 2 \cdot 10^{b}$, the detail channel
at level $b$ captures a fraction $\alpha_{b} \ge C(1-\varepsilon)^{B-1}$ of
total signal energy.
Empirically, the top~3 channels cover $\ge 97\%$ of energy for $B = 3, 4, 5$.

The remainder of this paper is organized as follows.
Section~\ref{sec:transform} defines $\varphi$-NTT and the convolution theorem on
$\ZB$.
Section~\ref{sec:wavelet} develops the hierarchical wavelet interpretation.
Section~\ref{sec:quasithm} states and sketches the proof of the quasi-theorem.
Section~\ref{sec:experiments} presents numerical experiments for $B = 1, \ldots, 5$.
Section~\ref{sec:discussion} discusses open problems, including the exact
orthogonality of $T/U$ projections and the determination of constants $C$ and
$\varepsilon$.
Section~\ref{sec:conclusion} concludes.

% ══════════════════════════════════════════════════════════════════════════════
\section{$\varphi$-NTT on $\mathbb{Z}_{10}^{B}$}
\label{sec:transform}

\subsection{Carry-Free Group Structure}

Let $B$ be a positive integer and $N = 10^{B}$.
We represent each index $n \in \{0, \ldots, N-1\}$ by its decimal digit expansion
\begin{equation}
  n = d_{0}(n) + 10\,d_{1}(n) + \cdots + 10^{B-1}\,d_{B-1}(n),
  \qquad d_{i}(n) \in \{0,\ldots,9\}.
\end{equation}
The \emph{carry-free group} $\ZB$ is the direct product
$\Zte \times \cdots \times \Zte$ ($B$ copies), with addition defined
digit-wise:
\begin{equation}
  (n \cfadd m)_{i} = \bigl(d_{i}(n) + d_{i}(m)\bigr) \bmod 10.
\end{equation}
This differs fundamentally from ordinary modular arithmetic on
$\mathbb{Z}/(10^{B})\mathbb{Z}$, where addition carries across digit boundaries.
The two groups coincide only at $B = 1$; for $B \ge 2$ they are non-isomorphic,
and the standard cyclic convolution theorem does not apply to $\ZB$.

Carry-free subtraction and negation are defined analogously:
\begin{align}
  \mathrm{cf\_sub}(n,m)_{i} &= \bigl(d_{i}(n) - d_{i}(m)\bigr) \bmod 10, \\
  \mathrm{cf\_neg}(n)_{i}   &= \bigl(-d_{i}(n)\bigr) \bmod 10.
\end{align}
All arithmetic in $\varphi$-NTT operates over the ring
$\Zphi = \mathbb{Z}[\sqrt{5}]$, where $\varphi = (1+\sqrt{5})/2$.
Elements are pairs $(a,b) \in \mathbb{Z}^{2}$ representing $a + b\varphi$, with
multiplication
\begin{equation}
  (a,b)\cdot(c,d) = (ac+bd,\; ad+bc+bd).
\end{equation}
This representation eliminates floating-point error entirely.

\subsection{Definition of $\varphi$-NTT}

For $B = 1$ (length $N = 10$), the transform decomposes the input
$x : \Zte \to \Zphi$ into two channels via the isomorphism
$\Zte \cong \mathbb{Z}_{2} \times \mathbb{Z}_{5}$ (Chinese Remainder Theorem):
\begin{align}
  T\text{-channel:} \quad
    X_{T}[k] &= \mathrm{n10\_T}(x)[k]
    \quad \text{($C_{5}$-twisted DFT projection)}, \label{eq:Tchannel}\\
  U\text{-channel:} \quad
    X_{U}[k] &= \mathrm{n10\_U}(x)[k]
    \quad \text{($\mathbb{Z}_{2}$ parity projection)}.  \label{eq:Uchannel}
\end{align}
The $C_{5}$-twisted DFT uses the basis elements
$\mathrm{cos5\_zphi}(m) \in \{(2,0),(-1,1),(0,-1)\}$---three distinct values in
$\Zphi$, with no irrational twiddle factors at runtime.
The $\mathbb{Z}_{2}$ projection uses the parity kernel
$U_{1}[k] \in \{(0,0),(\pm 1,0),(0,\pm 1)\}$.
Explicit tables are given in Appendix~\ref{app:basis}.

For general $B$, the transform is defined by a $B$-stage tensor product.
At each stage $b \in \{0,\ldots,B-1\}$, the input is projected onto either the
$T$-channel (low-pass, $C_{5}$ side) or the $U$-channel (high-pass,
$\mathbb{Z}_{2}$ side), yielding $2^{B}$ output channels labeled by
$s \in \{T,U\}^{B}$:
\begin{equation}
  \mathrm{bn\_forward\_flat}(x, B)
  \;\longrightarrow\;
  \bigl[\,\mathrm{ch}_{s} : s \in \{T,U\}^{B}\,\bigr],
  \quad \mathrm{ch}_{s} : \ZB \to \Zphi.
\end{equation}
The inverse transform $\mathrm{bn\_inverse\_flat}$ reconstructs $x$ exactly, with
scale factor $40^{B}$ (no convolution) or $80^{B} = 40^{B} \times 2^{B}$ (after
channel-domain convolution).
Both scale factors divide exactly in $\Zphi$, so reconstruction is lossless over
the integers.

A key structural property is that \emph{no inter-stage twiddle factors appear}:
the tensor-product theorem over $\ZB$ guarantees that all cross-stage phase
corrections vanish across all $B$ recursive stages.

\subsection{Convolution Theorem (Carry-Free)}

Define the carry-free convolution of $x, h : \ZB \to \Zphi$ as
\begin{equation}
  (x \star h)[n] = \sum_{m} h[m]\cdot x\bigl[\mathrm{cf\_sub}(n,m)\bigr],
  \qquad n \in \ZB.
\end{equation}

\begin{theorem}[Carry-Free Convolution]
\label{thm:conv}
Let $F_{B}$ denote the $\varphi$-NTT forward transform.
Then $F_{B}$ diagonalizes carry-free convolution on $\ZB$: the channel-domain
product
\[
  Y_{s}[k] = X_{s}[k] \cdot H_{s}[k]
  \quad \text{(with the $T/U$ cross-coupling rule)}
\]
recovers $x \star h$ upon inverse transform, up to scale factor $80^{B}$.
The $T/U$ cross-coupling at each stage is:
\begin{align}
  2\,Y_{T}[k] &= X_{T}[k]\,H_{T}[k]
               - \mathrm{SIN2} \cdot X_{U}[k]\,H_{U}[k], \\
  2\,Y_{U}[k] &= X_{T}[k]\,H_{U}[k]
               + X_{U}[k]\,H_{T}[k],
\end{align}
where $\mathrm{SIN2} = (3,-1) \in \Zphi \approx 1.382 = 4\sin^{2}(2\pi/10)$.
\end{theorem}

Theorem~\ref{thm:conv} has been verified for $B = 1, \ldots, 5$ against direct
computation.
Similarly, carry-free cross-correlation reduces to convolution with the
carry-free-reversed kernel $h_{\mathrm{rev}}[m] = h[\mathrm{cf\_neg}(m)]$.

% ══════════════════════════════════════════════════════════════════════════════
\section{Hierarchical Wavelet Interpretation}
\label{sec:wavelet}

The recursive $T/U$ decomposition induces a hierarchical structure on $\ZB$ that
parallels multiresolution analysis (MRA).

\subsection{$T/U$ as Projection-Like Operators}

At each stage $b \in \{0,\ldots,B-1\}$, the $\varphi$-NTT applies one of two
projection-like operators to digit $d_{b}$:
\begin{align*}
  T_{b} &: \text{low-pass side} \quad
    \text{(extracts the slowly-varying component across $d_{b} \in \{0,\ldots,9\}$)}, \\
  U_{b} &: \text{high-pass side} \quad
    \text{(extracts the alternating component $(-1)^{d_{b}}$)}.
\end{align*}
Together, $T_{b}$ and $U_{b}$ partition the information in digit $b$ in a manner
analogous to a two-channel filter bank.
We use the term \emph{projection-like} deliberately: while numerical experiments
confirm that energy is partitioned across channels without loss (exact
reconstruction holds for all $B = 1, \ldots, 5$), the strict algebraic
orthogonality $T_{b} \perp U_{b}$ has not yet been proved and is left as an open
problem (Section~\ref{sec:discussion}).

The $B$-stage tensor product applies these operators independently at each digit
position, yielding $2^{B}$ channels:
\begin{equation}
  \mathrm{ch}_{s} = \Bigl(\bigotimes_{b=0}^{B-1} P_{b}^{s_{b}}\Bigr)(x),
  \qquad s \in \{T,U\}^{B},
\end{equation}
where $P_{b}^{T} = T_{b}$ and $P_{b}^{U} = U_{b}$.

\subsection{Binary Tree Interpretation}

The label set $\{T,U\}^{B}$ is in natural bijection with the set of
root-to-leaf paths in a complete binary tree of depth $B$.
Reading the label from left (most significant digit, $b = B-1$) to right
(least significant, $b = 0$), the tree has the form:
\begin{center}
\begin{verbatim}
          root
        /       \
   T(d_{B-1})  U(d_{B-1})
     /   \       /   \
    T     U    T     U
   (d_{B-2})  ...
\end{verbatim}
\end{center}
The channel $TT\cdots T$ (all $T$, leftmost leaf) captures the coarsest
approximation; the channel $UU\cdots U$ (all $U$, rightmost leaf) captures the
finest detail---the checkerboard pattern across all digit positions simultaneously.

Table~\ref{tab:haar} contrasts the $\varphi$-NTT with Haar MRA\@.

\begin{table}[h]
\centering
\caption{Comparison of Haar MRA and $\varphi$-NTT.}
\label{tab:haar}
\begin{tabular}{lll}
\toprule
Property & Haar MRA & $\varphi$-NTT \\
\midrule
Signal length      & $2^{B}$           & $10^{B}$ \\
Channels           & $2^{B}$           & $2^{B}$ \\
Approximation coeff.& scaling function  & $TT\cdots T$ channel \\
Detail coeff.      & wavelet function   & $T\cdots T U_{b} T\cdots T$ channels \\
Arithmetic         & real ($\pm 1$ basis) & $\Zphi$ basis \\
Orthogonality      & exact             & conjectured \\
\bottomrule
\end{tabular}
\end{table}

\subsection{Energy Localization Mechanism}

Smooth signals vary slowly across all digit positions.
The $U_{b}$ operator measures the parity alternation $(-1)^{d_{b}}$ at digit $b$;
if $x$ changes little between even and odd values of $d_{b}$, the $U_{b}$ inner
product is small.
Accumulating this across $B$ stages, the $UU\cdots U$ channel is driven toward
zero, explaining the empirical observation (B = 3, 4, 5) that $UU\cdots U$
carries $<0.01\%$ of total energy for smooth signals.

Signals with a dominant period $P \approx 2\cdot 10^{b}$ exhibit strong
alternation specifically at digit $b$, while other digits vary slowly.
The $U_{b}$ projection then captures the bulk of the oscillatory energy, while
$T$ projections at all other stages pass the slowly varying envelope.
The result is that the channel $T\cdots T[U_{b}]T\cdots T$ (U only at position
$b$) becomes dominant.
For the test signal $\sin(2\pi n/100)$ with $P = 100 \approx 2\cdot 10^{1}$
($b=1$), this predicts the $TUT\cdots T$ channel as dominant---consistent with
all $B = 3, 4, 5$ experiments ($86\%$, $69\%$, $68\%$ of total energy,
respectively).

% ══════════════════════════════════════════════════════════════════════════════
\section{Spectral Concentration Quasi-Theorem}
\label{sec:quasithm}

We formalize the observed concentration phenomenon.

\subsection{Statement}

We work under two assumptions on the input signal
$x : \ZB \to \Zphi$ (real-valued, i.e., second component identically 0):

\begin{itemize}
\item[(A1)] \textbf{Digit-wise smoothness.}
  For each digit position $b' \ne b$, the variation of $x$ across $d_{b'}$ is
  small relative to the variation across $d_{b}$:
  \[
    |\mathrm{variation}(x, b')| \le \varepsilon \cdot |\mathrm{variation}(x, b)|,
    \qquad b' \ne b,
  \]
  for some constant $0 < \varepsilon < 1$.

\item[(A2)] \textbf{Dominant period.}
  The signal has a dominant period $P \approx 2\cdot 10^{b}$ for some
  $b \in \{0,\ldots,B-1\}$, so that $d_{b}$ is the primary source of oscillation.
\end{itemize}

\begin{theorem}[$\varphi$-NTT Spectral Concentration, Quasi-Theorem]
\label{thm:quasi}
Under assumptions \emph{(A1)} and \emph{(A2)}, there exist constants $C > 0$
and $0 < \varepsilon < 1$ (depending on the signal class) such that the energy
fraction captured by the level-$b$ detail channel satisfies
\begin{equation}
  \alpha_{b}
  \;=\;
  \frac{E\bigl[\mathrm{ch}_{T\cdots TU_{b}T\cdots T}\bigr]}{E_{\mathrm{total}}}
  \;\ge\;
  C \cdot (1-\varepsilon)^{B-1}.
\end{equation}
Furthermore, the top~3 channels (by energy) jointly satisfy
\begin{equation}
  \alpha_{b} + \alpha_{b,2} + \alpha_{b,3} \;\ge\; 1 - \delta
\end{equation}
for a small residual $\delta$, empirically $\delta < 0.03$ for $B = 3, 4, 5$.
\end{theorem}

We call this a \emph{quasi-theorem} because the constants $C$ and $\varepsilon$
are characterized implicitly through (A1)--(A2) rather than in closed form, and
because the strict algebraic orthogonality of the $T/U$ decomposition---while
consistent with all numerical evidence---remains unproved (see
Section~\ref{sec:discussion}).

\subsection{Interpretation}

The bound $C(1-\varepsilon)^{B-1}$ has a transparent structure:
$\varepsilon$ measures digit-level roughness---how much energy leaks from the
dominant digit $b$ to each of the other $B-1$ digits.
The factor $(1-\varepsilon)^{B-1}$ is the product of $B-1$ per-stage suppression
factors, explaining why the observed $\alpha_{b}$ decreases from $86\%$ ($B=3$)
to $68\%$ ($B=4,5$) as more stages accumulate leakage.
The constant $C$ absorbs the baseline energy ratio at the dominant stage itself.

The exponential decay in $B$ is a structural feature: with more digit stages,
more channels compete for energy.
The top-3 coverage ($\ge 97\%$) remains stable because the channels at levels
$b \pm 1$ absorb most of the leakage.

\subsection{Proof Sketch}

We sketch the structure of the argument; a complete algebraic proof is left for
future work.

\textbf{Step 1} (Digit smoothness $\Rightarrow$ $U$ energy bound).
By (A1), for each $b' \ne b$, the $U_{b'}$ projection has small inner product
with $x$, giving $E[U_{b'}] \le \varepsilon^{2} \cdot E[U_{b}]$ for $b' \ne b$.

\textbf{Step 2} (Dominant digit $\Rightarrow$ $U_{b}$ lower bound).
By (A2), $x$ oscillates with period $\approx 2\cdot 10^{b}$, correlating
strongly with the $\mathbb{Z}_{2}$ parity indicator $(-1)^{d_{b}}$.
This is precisely the kernel of $U_{b}$, yielding
$E[U_{b}] \ge C_{0}\cdot E_{\mathrm{total}}$ for some $C_{0} > 0$.

\textbf{Step 3} (Tensor product $\Rightarrow$ multiplicative separation).
Because the $B$-stage decomposition applies each operator independently per
digit, the channel $T\cdots TU_{b}T\cdots T$ receives energy from $U_{b}$ at
stage $b$ and $T$ at all other stages.
The $T$ projections at stages $b'\ne b$ pass through the slowly varying envelope
(factor $\approx 1$), while $U_{b'}$ channels at those stages are suppressed by
$\varepsilon$ per stage.

\textbf{Step 4} (Normalization $\Rightarrow$ $\alpha_{b}$ lower bound).
Summing the energy across all $2^{B}$ channels and normalizing yields
$\alpha_{b} \ge C(1-\varepsilon)^{B-1}$, where $C$ absorbs $C_{0}$ and the
$T$-stage pass-through factors.

\subsection{Empirical Alignment}

The quasi-theorem is consistent with all numerical experiments
($B = 3, 4, 5$; test signal $\sin(2\pi n/100)$, $P = 100 \approx 2\cdot 10^{1}$,
$b = 1$):

\begin{table}[h]
\centering
\caption{Empirical alignment with Quasi-Theorem~\ref{thm:quasi}.}
\label{tab:alignment}
\begin{tabular}{cll}
\toprule
$B$ & $\alpha_{b}$ (observed) & Lower bound structure \\
\midrule
3 & 0.864 & $C(1-\varepsilon)^{2}$ with $\varepsilon \approx 0.07$ \\
4 & 0.687 & $C(1-\varepsilon)^{3}$ \\
5 & 0.682 & $C(1-\varepsilon)^{4}$ \\
\bottomrule
\end{tabular}
\end{table}

In all cases, the top~3 channels jointly cover $\ge 97\%$ of total energy, and
the $UU\cdots U$ channel carries $<0.01\%$ in all experiments.
The values of $C$ and $\varepsilon$ are not fitted here; the table shows only
that the observed decay is consistent with the bound structure
$C(1-\varepsilon)^{B-1}$.

% ══════════════════════════════════════════════════════════════════════════════
\section{Numerical Experiments}
\label{sec:experiments}

\subsection{Experimental Setup}

All experiments use the reference implementation \texttt{phi\_carry\_free.py},
operating entirely in exact integer arithmetic over $\Zphi$.
No floating-point operations are involved in the transform itself;
\texttt{zval()} (evaluation to float) is used only for energy measurement in
post-processing.

We test $B = 1, \ldots, 5$ (signal lengths $N = 10$ to $100{,}000$) with a
low-frequency test signal
\[
  x[n] = \sin(2\pi n/100) + \sin(2\pi n/P_{2}) + \mathrm{noise},
\]
where $P_{2}$ is a secondary period scaled to $N$ and the noise is a small
integer perturbation.
This signal class satisfies assumptions (A1)--(A2) of Section~\ref{sec:quasithm}
with dominant period $P = 100 \approx 2\cdot 10^{1}$ ($b = 1$).

Correctness is verified in two ways: (i) direct comparison of
\texttt{phi\_conv\_carry\_free} against the reference $O(N^{2})$ carry-free
convolution, and (ii) round-trip test with a delta kernel $h = \delta_{0}$,
confirming exact reconstruction for all $B = 1, \ldots, 5$.

\subsection{Channel Energy Distribution}

Table~\ref{tab:energy} shows the energy fraction of the top-3 channels and the
$UU\cdots U$ channel for $B = 3, 4, 5$.
All fractions are computed as
\[
  E[\mathrm{ch}] / E_{\mathrm{total}},
  \qquad
  E[\mathrm{ch}] = \frac{1}{N}\sum_{k} \bigl(\mathrm{zval}(\mathrm{ch}[k])\bigr)^{2}.
\]

\begin{table}[h]
\centering
\caption{Channel energy distribution (low-frequency test signal).}
\label{tab:energy}
\begin{tabular}{clrclrclrrrr}
\toprule
$B$ & \#1 ch & $\alpha_{1}$ & \#2 ch & $\alpha_{2}$ & \#3 ch & $\alpha_{3}$
    & Top-3 & $UU{\cdots}U$ \\
\midrule
3 & TUT   & 86.4\% & TTT   & 11.1\% & UTT  & 2.3\%  & 99.8\% & $<0.01\%$ \\
4 & TUTT  & 68.7\% & TTUT  & 17.7\% & TTTT & 11.7\% & 98.1\% & $<0.01\%$ \\
5 & TUTTT & 68.2\% & TTTUT & 17.7\% & TTTTT& 11.7\% & 97.6\% & $<0.01\%$ \\
\bottomrule
\end{tabular}
\end{table}

Three observations are immediate.
First, the dominant channel is consistently of the form $T\cdots TU_{1}T\cdots T$
(U only at digit $b=1$), in agreement with the Section~\ref{sec:wavelet}
prediction for $P \approx 2\cdot 10^{1}$.
Second, the top-3 channels jointly cover $\ge 97\%$ of total energy across all
tested values of $B$, consistent with the $\delta < 0.03$ claim of
Theorem~\ref{thm:quasi}.
Third, the $UU\cdots U$ channel carries negligible energy ($<0.01\%$) in all
cases, consistent with the checkerboard suppression argument of
Section~\ref{sec:wavelet}.

\begin{table}[h]
\centering
\caption{SNR vs.\ retained channels (low-frequency signal, $B = 3, 4, 5$).}
\label{tab:snr}
\begin{tabular}{ccccc}
\toprule
$B$ & 1 ch & 2 ch & 4 ch & All ch \\
\midrule
3 & 9.5 dB  & 15.4 dB & 28.5 dB & $\infty$ (exact) \\
4 & 5.3 dB  & 9.5 dB  & 20.5 dB & $\infty$ (exact) \\
5 & 5.2 dB  & 9.3 dB  & 18.5 dB & $\infty$ (exact) \\
\bottomrule
\end{tabular}
\end{table}

The ``$\infty$ (exact)'' entries confirm lossless reconstruction when all
channels are retained.

\subsection{Reconstruction Accuracy}

Exact reconstruction is guaranteed by the algebraic structure of $\varphi$-NTT,
not by numerical precision.
The scale factors $40^{B}$ (round-trip) and $80^{B}$ (after convolution) divide
exactly in $\Zphi$, as verified by assertion checks for all $B = 1, \ldots, 5$:
\begin{verbatim}
assert v[0] % scale == 0 and v[1] % scale == 0   # holds for all v
\end{verbatim}
The delta-kernel test \texttt{phi\_conv\_carry\_free(x, $\delta_{0}$, B) == x}
passes for all tested inputs and all $B = 1, \ldots, 5$, confirming
Theorem~\ref{thm:conv}.

\subsection{Computational Structure}

The $\varphi$-NTT avoids two sources of complexity common in FFT implementations.
First, there are no inter-stage twiddle factors: all phase corrections vanish by
the tensor-product theorem over $\ZB$.
Second, all arithmetic is integer-valued; the only divisions are the final
scale-factor normalizations, which are exact.
The transform operates on length-$10^{B}$ signals with $2^{B}$ output channels
in $O(N \cdot B)$ integer operations per stage.
Detailed complexity analysis and comparison with FFT are reserved for future work.

% ══════════════════════════════════════════════════════════════════════════════
\section{Discussion}
\label{sec:discussion}

\subsection{Open Problems}

Three problems remain open.

\textbf{(P1) Orthogonality of $T/U$ projections.}
The numerical experiments confirm that energy is partitioned across channels
without loss and that exact reconstruction holds for all tested $B$.
However, the strict algebraic orthogonality $T_{b} \perp U_{b}$ over $\Zphi$ has
not been proved.
Establishing this would upgrade Theorem~\ref{thm:quasi} to a full theorem, with
$C$ and $\varepsilon$ determined by the projection geometry.

\textbf{(P2) Closed-form determination of $C$ and $\varepsilon$.}
The constants $C$ and $\varepsilon$ in Theorem~\ref{thm:quasi} are currently
characterized implicitly through (A1)--(A2).
Deriving explicit expressions---presumably in terms of the signal period $P$,
the digit base 10, and the $\Zphi$ basis elements---would make the spectral
concentration bound fully quantitative and signal-class independent.

\textbf{(P3) Connection to carry-based convolution (overlap-add).}
The present work is restricted to carry-free convolution on $\ZB$.
A natural extension is to bridge this to ordinary cyclic convolution on
$\mathbb{Z}/(10^{B})\mathbb{Z}$ via an overlap-add scheme, enabling
$\varphi$-NTT-based filtering of standard integer sequences.
The group non-isomorphism $\ZB \not\cong \mathbb{Z}/(10^{B})\mathbb{Z}$
(for $B \ge 2$) means this bridge is non-trivial and requires a separate
treatment.

\subsection{Relation to Existing Frameworks}

\textbf{vs.\ FFT.}
Standard FFT operates on the cyclic group $\mathbb{Z}/(N)\mathbb{Z}$ with
complex twiddle factors, accumulating floating-point error.
$\varphi$-NTT operates on $\ZB$ with coefficients in $\Zphi$, achieving exact
integer arithmetic.
The two transforms diagonalize different convolution operations and are not
directly comparable in terms of computational cost; FFT runs in $O(N\log N)$
while the complexity of $\varphi$-NTT is $O(NB)$ with $N = 10^{B}$.

\textbf{vs.\ Standard NTTs.}
Galois-field NTTs (e.g., used in lattice-based cryptography) achieve exact
integer arithmetic by working modulo a prime $p$.
$\varphi$-NTT works over $\Zphi$ without modular reduction, preserving the full
digit structure of $\ZB$.
The two frameworks address different algebraic settings and are complementary.

\textbf{vs.\ Haar Wavelets.}
Haar MRA on length-$2^{B}$ signals uses $\pm 1$ arithmetic and achieves
exact orthogonality.
$\varphi$-NTT on length-$10^{B}$ signals uses $\Zphi$ arithmetic and conjectured
quasi-orthogonality.
The structural analogy (Section~\ref{sec:wavelet}) is precise; the arithmetic
and group structure differ.

\subsection{Limitations}

The current work has the following limitations.
The quasi-theorem (Theorem~\ref{thm:quasi}) is stated for real-valued signals
satisfying (A1)--(A2); extension to complex-valued or high-bandwidth signals
remains open.
The computational complexity $O(NB)$ has not been compared to FFT implementations
for the same problem size.
The carry-free convolution studied here does not directly apply to standard
digital filtering tasks, which rely on ordinary cyclic convolution (Problem~P3).

% ══════════════════════════════════════════════════════════════════════════════
\section{Conclusion}
\label{sec:conclusion}

We have introduced $\varphi$-NTT, a transform defined over the carry-free direct
product group $\mathbb{Z}_{10}^{B}$ with coefficients in the golden-ratio integer
ring $\Zphi$.
The three main contributions are as follows.

First, we establish a carry-free convolution theorem
(Theorem~\ref{thm:conv}): $\varphi$-NTT diagonalizes carry-free convolution
on $\ZB$, with exact integer arithmetic and no floating-point error.

Second, we develop a hierarchical wavelet interpretation
(Section~\ref{sec:wavelet}): the $B$-stage $T/U$ decomposition is structurally
analogous to Haar MRA, producing $2^{B}$ channels arranged in a complete binary
tree of depth $B$.

Third, we state a Spectral Concentration Quasi-Theorem
(Theorem~\ref{thm:quasi}): for signals with dominant period
$P \approx 2\cdot 10^{b}$, the level-$b$ detail channel captures energy fraction
$\alpha_{b} \ge C(1-\varepsilon)^{B-1}$, and the top~3 channels jointly cover
$\ge 97\%$ of total energy for $B = 3, 4, 5$.

Open problems include the algebraic proof of $T/U$ orthogonality (P1),
closed-form determination of $C$ and $\varepsilon$ (P2), and the connection to
standard carry-based convolution via overlap-add (P3).
We hope this work opens a productive line of inquiry at the intersection of
carry-free arithmetic, wavelet theory, and exact integer transforms.

% ══════════════════════════════════════════════════════════════════════════════
\appendix

\section{Basis Tables for $B=1$}
\label{app:basis}

\subsection{$C_{5}$-Twisted DFT Basis (T-channel)}

The $T$-channel basis elements $\mathrm{cos5\_zphi}(m) \in \Zphi$ for
$m \in \{0,1,2,3,4\}$ (one representative per $C_{5}$ orbit) take exactly three
distinct values:
\[
  \mathrm{cos5\_zphi}(m) \in
  \{(2,0),\;(-1,1),\;(0,-1)\} \subset \Zphi,
\]
corresponding to $2\cos(0) = 2$, $2\cos(2\pi/5) = \varphi - 1$, and
$2\cos(4\pi/5) = -\varphi^{-1}$ respectively, expressed as elements of $\Zphi$.
No irrational twiddle factors appear at runtime.

\subsection{$\mathbb{Z}_{2}$ Parity Kernel (U-channel)}

The $U$-channel parity kernel $U_{1}[k]$ for $k \in \{0,\ldots,9\}$ takes
values in $\{(0,0), (\pm 1,0), (0,\pm 1)\} \subset \Zphi$, corresponding to
the parity indicator $(-1)^{k}$ lifted into $\Zphi$ via the even/odd split
of $\Zte \cong \mathbb{Z}_{2} \times \mathbb{Z}_{5}$.

% ── Appendix B ──────────────────────────────────────────────────────────────
\section{Permutation \texttt{perm10}}
\label{app:perm10}

The permutation \texttt{perm10} reindexes $\{0,\ldots,9\}$ to align the CRT
decomposition $\Zte \cong \mathbb{Z}_{2} \times \mathbb{Z}_{5}$ with the natural
digit order.
Explicitly, \texttt{perm10} maps $k \mapsto \sigma(k)$ where
$\sigma = (0, 5, 2, 7, 4, 9, 6, 1, 8, 3)$ (0-indexed).
This permutation is its own inverse up to sign, and all $\varphi$-NTT stages
apply it before the $C_{5}$ DFT step.
The full derivation is given in the reference implementation
\texttt{phi\_carry\_free.py}.

% ── Appendix C ──────────────────────────────────────────────────────────────
\section{$80^{B}$ Scale Factor Decomposition}
\label{app:scale}

The scale factor $80^{B}$ arising from a full forward--inverse cycle with
carry-free convolution decomposes as
\[
  80^{B} = 40^{B} \times 2^{B},
\]
where $40^{B}$ arises from the $B$-stage forward--inverse round trip (each stage
contributing a factor of 40) and the additional $2^{B}$ arises from the
$T/U$ cross-coupling normalization at each of the $B$ stages.
Both factors are exact divisors in $\Zphi$ (no remainder), ensuring lossless
integer reconstruction.
The complete verification that $80^{B}$ divides all output elements exactly is
implemented in the assertion checks of \texttt{phi\_carry\_free.py} and holds for
all tested $B = 1, \ldots, 5$.

% ══════════════════════════════════════════════════════════════════════════════
\end{document}
